\documentclass[11pt,fleqn]{article}

\setlength {\topmargin} {-.15in}
\setlength {\textheight} {8.6in}

\usepackage{amsmath}
\usepackage{amssymb}
\usepackage{amsthm}
\usepackage{color}
%\usepackage{enumitem,xcolor}
\usepackage{fullpage}
\usepackage{enumerate}

\renewcommand{\labelenumi}{\theenumi.}
\renewcommand{\labelenumii}{\theenumii.}
\renewcommand{\labelenumiii}{\theenumiii.}
\newcommand{\be}{\begin{enumerate}}
\newcommand{\ee}{\end{enumerate}}
\newcommand{\bi}{\begin{itemize}}
\newcommand{\ei}{\end{itemize}}
\newcommand{\bc}{\begin{center}}
\newcommand{\ec}{\end{center}}
\newcommand{\bsp}{\begin{sloppypar}}
\newcommand{\esp}{\end{sloppypar}}
\newcommand{\mname}[1]{\mbox{\sf #1}}
\newcommand{\pnote}[1]{{\langle \text{#1} \rangle}}
\newcommand{\sB}{\mbox{$\cal B$}}
\newcommand{\sC}{\mbox{$\cal C$}}
\newcommand{\sF}{\mbox{$\cal F$}}
\newcommand{\sP}{\mbox{$\cal P$}}
\newcommand{\subfun}{\sqsubseteq}
\ifdefined \And 
\renewcommand{\And}{\wedge}
\else
\newcommand{\And}{\wedge}
\fi

\begin{document}

\begin{center}

  {\large \textbf{COMPSCI/SFWRENG 2FA3}}\\[2mm]
  {\large \textbf{Discrete Mathematics with Applications II}}\\[2mm]
  {\large \textbf{Winter 2020}}\\[8mm]
  {\huge \textbf{Assignment 3}}\\[6mm]
  {\large \textbf{Dr.~William M. Farmer}}\\[2mm]
  {\large \textbf{McMaster University}}\\[6mm]
  {\large Revised: February 6, 2020}

\end{center}

\medskip

Assignment 3 consists of four problems.  You must write your solutions
to the problems using LaTeX.

Please submit Assignment~3 as two files,
\texttt{Assignment\_3\_\emph{YourMacID}.tex} and
\texttt{Assignment\_3\_\emph{YourMacID}.pdf}, to the Assignment~3
folder on Avenue under Assessments/Assignments.
\texttt{\emph{YourMacID}} must be your personal MacID (written without
capitalization).  The \texttt{Assignment\_3\_\emph{YourMacID}.tex}
file is a copy of the LaTeX source file for this assignment
(\texttt{Assignment\_3.tex} found on Avenue under
Contents/Assignments) with your solution entered after each problem.
The \texttt{Assignment\_3\_\emph{YourMacID}.pdf} is the PDF output
produced by executing

\begin{itemize}

  \item[] \texttt{pdflatex Assignment\_3\_\emph{YourMacID}}

\end{itemize}

This assignment is due \textbf{Sunday, February 9, 2020 before
  midnight.}  You are allow to submit the assignment multiple times,
but only the last submission will be marked.  \textbf{Late submissions
  and files that are not named exactly as specified above will not be
  accepted!}  It is suggested that you submit your preliminary
\texttt{Assignment\_3\_\emph{YourMacID}.tex} and
\texttt{Assignment\_3\_\emph{YourMacID}.pdf} files well before the
deadline so that your mark is not zero if, e.g., your computer fails
at 11:50 PM on February 9.

\textbf{Although you are allowed to receive help from the
  instructional staff and other students, your submission must be your
  own work.  Copying will be treated as academic dishonesty! If any of
  the ideas used in your submission were obtained from other students
  or sources outside of the lectures and tutorials, you must
  acknowledge where or from whom these ideas were obtained.}

\newpage

\subsection*{Background}

Let $\Sigma = (\sB,\sC,\sF,\sP,\tau)$ be a finite signature of MSFOL,
$F_{\Sigma}$ be the set of $\Sigma$-formulas, and $A \in F_{\Sigma}$.
Recall that the members of $F_{\Sigma}$ are certain strings of
symbols.  A \emph{subformula} of $A$ is a $B \in F_{\Sigma}$ such that
$B$ is a substring of $A$.  For example, let $A$ be the formula $((0 =
2) \And (3 \mid 4))$, i.e., $A$ is the string ``$((0 = 2) \And (3 \mid
4))$''.  Then ``$(0 = 2)$'', ``$(3 \mid 4)$'', and ``$((0 = 2) \And (3
\mid 4))$'' are the subformulas of $A$, and ``$(0 = {}$'' and
``$\And$'' are two substrings of $A$ that are not subformulas of $A$.

A function $f : A \rightarrow B$ is \emph{total} if it is defined on
\emph{all} members of $A$.  A function $f : A \rightarrow B$ is a
\emph{partial} if it is be undefined on \emph{some} members of $A$.
For example, the square root function $\sqrt{\cdot} : \mathbb{R}
\rightarrow \mathbb{R}$ is a partial function since $\sqrt{r}$ is
undefined for all $r \in \mathbb{R}$ with $r < 0$. If $f,g : A
\rightarrow B$ are partial or total functions, then $f$ is a
\emph{subfunction} of $g$, written $f \subfun g$, if the domain $D_f$
of $f$ is a subset of the domain of $g$ and, for all $x \in D_f$,
$f(x) = g(x)$.  In other words, $f$ is a subfunction of $g$ if $g(a)$
is defined and $f(a) = g(a)$ whenever $f(a)$ is defined.

\subsection*{Problems}

\be

  \item\textbf{[10 points]} Let $\mname{subformulas} : F_{\Sigma}
    \rightarrow \sP(F_{\Sigma})$ be the function that maps a formula
    $A \in F_{\Sigma}$ to the set of subformulas of $A$.  Define
    $\mname{subformulas}$ by structural recursion using pattern
    matching.

  \textcolor{blue}{\textbf{Name: Hishmat Salehi \\ MacID: salehh6 \\ Date: \today}}

  \textcolor{blue}{\textbf{Problem 1}}

\begin{enumerate}[1)]
\item Let $E = t_{1} = t_{2}$ where $t_{1}, t_{2} \in \Sigma_{terms}$ and E $\in F_{\Sigma}$, then\\
\indent \indent \mname{subformulas('E') = \{'E'\}}

\item Let $P = p(t_{1},\dots, t_{n})$ where $t_{1},\dots, t_{n} \in \Sigma_{terms}$ and P $\in F_{\Sigma}$, then\\
\indent \indent \mname{subformulas('P') = \{'P'\}}

\item Let  $A, \lnot A \in F_{\Sigma}$, then\\
\indent \indent \mname{subformulas('A') = $\{'\lnot A'\}$ $\cup$ subformulas($A$)} 

\item Let  $B, C$ and $B \implies C \in F_{\Sigma}$, then\\
\indent \indent \mname{subformulas($'B \implies C'$) = $\{'B \implies C'\}$ $\cup$ subformulas($'B'$) $\cup$ subformulas($'C'$)} 

\item Let  $x \in \mathcal{V}, \alpha \in \beta$ and D $\in F_{\Sigma}$, and $\forall x : \alpha$. $D$, then\\
\indent \indent \mname{subformulas($'\forall x : \alpha$. $D'$) = $\{'\forall x : \alpha$. $D'\}$ $\cup$ subformulas($'D'$)} 

\end{enumerate}

\newpage

  \item\textbf{[10 points]} Suppose $F$ is the set of partial and
    total functions $f : \mathbb{N} \rightarrow \mathbb{N}$.

  \be

    \item Show that $(F,\subfun)$ is a weak partial order but not a
      weak total order.

    \item Describe the set of minimal elements of $(F,\subfun)$.


    \item Describe the set of maximal elements of $(F,\subfun)$.


    \item Does $(F,\subfun)$ have a minimum element?  If so, what is
      it?


    \item Does $(F,\subfun)$ have a maximum element?  If so, what is
      it?

  \ee

  \textcolor{blue}{\textbf{Name: Hishmat Salehi \\ MacID: salehh6 \\ Date: \today}}

  \textcolor{blue}{\textbf{Problem 2}}

\begin{enumerate}[a)]
\item
\begin{proof}
	If $(F,\subfun)$ is reflexive, antisymmetric, and transitive, then it is a weak partial order.
	We already know that F is a set of partial and total functions. Definitions for a weak partial order, for $(F,\subfun)$, are given below:
	\begin{align*}
		\forall f \in F.~		& f \subfun f 									& \pnote{reflexive}\\
		\forall f,g	\in F.~		&  f \subfun g \wedge  g \subfun f \Rightarrow f=g 		& \pnote{antisymmetric}\\
		\forall f,g,h \in F.~ 	& f \subfun g \wedge  g \subfun h \Rightarrow f \subfun h 	& \pnote{transitive}
	\end{align*}
	
	We also have the definition of $(\subfun)$:
	\begin{align*}
		f \subfun g \equiv D_f \subseteq D_g \wedge \forall x \in D_f. ~ f(x) = g(x)
	\end{align*}
	\medskip
	\textbf{Reflexivity of $(F,\subfun)$:} For any $f \in F$ and $f : \mathbb{N} \rightarrow \mathbb{N}$ we see that:

	Since domain $\mathbb{N}$ is a subset of $\mathbb{N}$ and $\forall x \in \mathbb{N}$. f(x) = f(x), by definition of $(\subfun)$: \\
~~ $f \subfun f \equiv \mathbb{N} \subseteq \mathbb{N} \wedge \forall x \in \mathbb{N}. ~ f(x) = f(x) \equiv True$

	Therefore $\forall f \in F.~ f \subfun f$ holds.
	This shows that $(F,\subfun)$ is reflexive.

	\medskip
	\textbf{Antisymmetry of $(F,\subfun)$:} For any $f, g \in F$, where $f : \mathbb{N} \rightarrow \mathbb{N}$ and $g: \mathbb{N} \rightarrow \mathbb{N}$ we see that:
	\begin{align*}
				& \forall f,g	\in F.~  f \subfun g \wedge  g \subfun f 	  			&\pnote{By definition of $(\subfun)$}\\
	\Rightarrow~	& \forall f,g	\in F.~  \mathbb{N} \subseteq \mathbb{N} \wedge \forall x_1 \in \mathbb{N}. ~ f(x_1) = g(x_1) \wedge \forall x_2 \in \mathbb{N}. ~ g(x_2) = f(x_2) \\	 & \pnote{This means Domain of f and g are the same and $f = g$}\\	
	\Rightarrow~	& f = g															&
	\end{align*}
	By transitivity of $\Rightarrow$, $\forall f,g	\in F.~ f \subfun g \wedge  g \subfun f \Rightarrow f=g$.
	This shows that $(F,\subfun)$ is antisymmetric.

	\medskip
	\textbf{Transitivity of $(F,\subfun)$:} For any $f,g,h \in G$, where $f : \mathbb{N} \rightarrow \mathbb{N}$, $g: \mathbb{N} \rightarrow \mathbb{N}$ and $h: \mathbb{N} \rightarrow \mathbb{N}$ we see that:
	\begin{align*}
						& \forall f,g,h 	\in F.~ f \subfun g \wedge  g \subfun h		&\pnote{By definition of $(\subfun)$}\\
		\Rightarrow~	& \forall f,g,h	\in F.~  \mathbb{N} \subseteq \mathbb{N} \wedge \forall x_1 \in \mathbb{N}. ~ f(x_1) = g(x_1) \wedge \forall x_2 \in \mathbb{N}. ~ g(x_2) = h(x_2) \\	 & \pnote{Given $ f(x_1) = g(x_1)$ and $g(x_2) = h(x_2)$ for all $x_1, x_2$ $\in \mathbb{N}$}\\	
		\Rightarrow~	& \forall f,h	\in F.~  \forall x \in \mathbb{N}. ~ f(x) = h(x)	 & \pnote{By definition of $(\subfun)$}\\	
		\Rightarrow~	& f \subfun h
	\end{align*}
	By transitivity of $\forall f,g,h \in F.~ f \subfun g \wedge  g \subfun h \Rightarrow f \subfun h$.
	This shows that $(F,\subfun)$ is transitive.

	\medskip
	\textbf{Totality of $(F,\subfun)$:} Will will prove this by conterexample. For any $f, g \in F$, where $f : \mathbb{N} \rightarrow \mathbb{N}$ and $g: \mathbb{N} \rightarrow \mathbb{N}$ we see that:\\
If we pick f to be $f(x) = x+1$ and $g(x) = x*2$ then $f(2) = 3 \neq g(2) = 4$. Therefore $\lnot (\forall f,g \in F.~ f \subfun g \lor  g \subfun f)$.
	
	\medskip
	Therefore $(F,\subfun)$ is a weak partial order and not weak total order.
\end{proof}

\item All the partial functions of $(F,\subfun)$ are minimal elements. Also, if A is a subfunction of all the functions of a subset of $(F,\subfun)$, then A is a minimal element of $(F,\subfun)$.

\item All the total functions of $(F,\subfun)$ are maximal elements. Also, if all the functions of a subset of $(F,\subfun)$ are subfunctions of A, then A is a maximal element of $(F,\subfun)$.

\item Yes, the minimum element of $(F,\subfun)$ is a function that is partial, subfunction of all the functions in $(F,\subfun)$ and it's output is always the same element. For example, a function that always returns 0.

\item No, the $(F,\subfun)$ doesn't have a maximum element.

\end{enumerate}

\ee

\end{document}


